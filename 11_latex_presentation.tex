% Iniziamo l anostra presentazione con la funzione \documentclass{beamer}
\documentclass{beamer} 
\usepackage{graphicx} % Installiamo il pacchetto necessario per inserire immagini

% \usepackage{listings} Questo pacchetto ci permette di inserire parti di codice salvate

% Cercando "Beamer themes and colors" possiamo vedere i temi a sinistra i colori sulla riga in alto 
\usetheme{Hannover} % Diamo la struttura allla presentazione 
\usecolortheme{crane} % Diamo il colore alla presentazione, anche {dove} è carino

%Procediamo con la nostra presentazione aggiungendo titolo, relatore e data
\title{Presentazione di Latex}
\author{Beatrice Viola} 
\date{May 2024}

\begin{document} %Dopo ogni begin va sempre un end

\maketitle

% A cosa serve??????
\AtBeginSection[]
{
\begin{frame}{Outline}
\tableofcontents[currentsection]
\end{frame}
}

\section{Introduction}

% Per passare ad una nuova slide basta inserire \begin{frame}
\begin{frame}{My first slide} % Il titolo tra parentesi graffe appare in alto nella slide
    Here i am inserting my text
\end{frame}

% In un'altra slide inseriamo un elenco puntato
\begin{frame}{Itemize}
the presentation will be on:
    \begin{itemize} % Con questa funzione creiamo l'elenco puntato
        \item my first film
        \item \pause my second 
        \item \pause my third film
    \end{itemize}
\end{frame}
%con la funzione \pause usata prima gli ultimi due punti dell'elenco mi appaiono in altre slide, un po' come una sorta di animazione

%Cambiamo la dimensione del testo
\begin{frame}{Text dimension} 
    \scriptsize{here i am inserting my text}
\end{frame}

% Nuova slide
\begin{frame}{4} 
% I due backslash \\ servono per andare a capo
    \textit{hello}\\ %testo in corsivo
    \textbf{How are you} \\ %testo in grassetto
    \bigskip % Lasciamo uno spazio ampio tra una frase e l'altra
\end{frame}

\section{Formulas}

%Facciamo una formula matematica
\begin{frame}{Algotitms used} %Possiamo trovare le varie funzioni per le formule online
In this work I used the following formula:
\bigskip
\begin{equation}
    \delta = \sqrt{\frac{\displaystyle\sum_{i=1}^{N}{(x - \mu)^2}}{N}}
    % \delta = simbolo per la standard deviation
    % \sqrt = radice quadrata, inserisco tutto il resto della formula tra le sue parentesi
    % \frac = frazione, la prima parentesi graffa è il numeratore e la seconda il denominatore
    % \displaystyle = mi mette la i=1 sotto il simbolo sommatoria e la N sopra il simbolo sommatoria
    % \sum = mette il simbolo della sommatoria che va da 1 a N
    % \mu = il simbolo della lettera greca mi
\end{equation}
    
\end{frame}

%Nuova sezione dei risultati
\section{Results} 

\begin{frame}{Main results}

%Inseriamo una figura usando i tre puntini in alto su Overleaf
\begin{figure} 
    \centering % Mettiamo l'immagine al centro
    \includegraphics[width=0.9\linewidth]{apodemus.png} % 0.9 è una buona dimensione
   % \caption{} la togliamo che non si usa di solito nelle presentazioni
   % \label{fig:enter-label} % Anche la label nelle presentazioni non serve
\end{figure}


% Inseriamo 2 immagini una accanto all'altra
\begin{figure} 
    \centering 
    \includegraphics[width=0.4\linewidth]{apodemus.png} %in questo caso la dimensione è + piccola perché ho 2 immagini vicine
    \pause \includegraphics[width=0.4\linewidth]{apodemus.png} 
\end{figure} 
\end{frame}

% Mettiamo 4 immagini, due sopra e 2 sotto, quindi 2 righe e 2 colonne
\begin{frame}{4 figure}
\begin{figure} % non lo scrivi a mano, ma dai puntini in alto a dx fai instert image
    \centering % mi centra il grafico
    \includegraphics[width=0.3\linewidth]{apodemus.png} 
    \includegraphics[width=0.3\linewidth]{apodemus.png} \\ % vado a capo
    \includegraphics[width=0.3\linewidth]{apodemus.png} 
    \includegraphics[width=0.3\linewidth]{apodemus.png} \\ vado a capo
\end{figure}

% Inseriamo un testo che ci dica da dove l'immagine è stata presa
\bigskip % Lasciamo uno spazio grande
\centering % Mettiamo il testo centrale
\scriptsize{Viola et al. (2024)} % Se uso delle citazioni le metto tutte con questa funzione????
\end{frame}

% Mettiamo il nostro testo e l'immagine vicini
\begin{frame}{Columns} 

    \begin{columns}
% Creiamo la prima colonna
    \begin{column}{width=0.5\textwidth}
        \includegraphics[width=0.4\linewidth]{apodemus.png}       
    \end{column}
% creo la seconda
    \begin{column}{width=0.5\textwidth}
        Inseriamo un piccolo testo
    \end{column}
    \end{columns}
\end{frame}

\end{document}
