\documentclass{beamer} % la classe per le presentazioni
\usepackage{graphicx} % Required for inserting images

% \usepackage{listings} permette di inserire pezzi di codice salvati

% cerca : Beamer themes and colors, colonna a sx hai i theme e in riga in alto i colori
\usetheme{Hannover} % struttura della presentazione {Dresden è carina}
\usecolortheme{crane} % colore della presentazione
% duccio usa {Frankfurt} e {dove} , molto pulito

\title{Presentazione di Latex}
\author{Filippo Paganelli} % relatore
\date{May 2024}

\begin{document}

\maketitle

% dopo il \maketitle, ad ogni nuova sezione fai queste cose:
\AtBeginSection[]
{
\begin{frame}{Outline}
\tableofcontents[currentsection]
\end{frame}
}

\section{Introduction}

% le slide si chiamano frame
\begin{frame}{My first slide} % titolo in cima
    here i am inserting my text
\end{frame}

% meglio evitare inserire tanto testo, piuttosto elenchi puntati per inserire info
\begin{frame}{Itemize}
the presentation will be on:
    \begin{itemize} % elenco puntato
        \item my first film
        \item \pause my second % con \pause mi appaiono nella slide sucessiva (è tipo un'animazione un po' più grezza)
        \item \pause bla bla bla
    \end{itemize}
\end{frame}

% cerchi su G le funzioni per cambiare la formattazione del testo (\tiny, \huge...)
\begin{frame}{Text dimension} % cambiare dimensione
    \scriptsize{here i am inserting my text}
\end{frame}

\begin{frame}{4} 
    \textit{dajeeee}\\ % per andare a capo \\
    \textbf{caio} bello\\
    \bigskip % per lasciare spazio più ampio tra una frase e l'altra
    volevo scrivere \textbf{ciao} % è bene scrivere le parole principali in grassetto
\end{frame}

\section{formulas}

\begin{frame}{Algotitms used} % cerca su G: latex math wiki
In this work I used the following formula:
\bigskip
\begin{equation}
    \delta = \sqrt{\frac{\displaystyle\sum_{i=1}^{N}{(x - \mu)^2}}{N}}
    % \frac : il primo è il num e la seconda {} è il denom
    % \sum : mi mette il simbolo di sommatoria, se voglio mettere i=1 fino ad N, metto _{}^{}. se prima scrivo \displaystyle mi mette i e N sopra e sotto il simbolo 
    % \sqrt : square root
    % \mu : mi mette il simbolo
    % \delta : simbolo della sd
\end{equation}
    
\end{frame}

\section{Results} % sezione dei risultati

\begin{frame}{main results}

\begin{figure} % non lo scrivi a mano, ma dai puntini in alto a dx fai instert image
    \centering % mi centra il grafico
    \includegraphics[width=0.8\linewidth]{Screenshot 2024-05-08 alle 17.37.06.png} % 0.8 perchè se cambi un attimo lo stile della pres potrebbe sminchiare
   % \caption{negative correlation} solitamente la caption non si usa nelle presentazioni
    \label{fig:enter-label} % anche questa non serve
\end{figure}
    
\end{frame}

% voglio inserire 2 immagini accanto
\begin{frame}{main results}
\begin{figure} % non lo scrivi a mano, ma dai puntini in alto a dx fai instert image
    \centering % mi centra il grafico
    \includegraphics[width=0.4\linewidth]{Screenshot 2024-05-08 alle 17.37.06.png} 
    \pause \includegraphics[width=0.4\linewidth]{Screenshot 2024-05-08 alle 17.37.06.png} 
\end{figure} % le lascio nello stesso \begin

% per metterle una sopra l'altra faccio 2 begin
\end{frame}

% 4 immagini
\begin{frame}{4 figure}
\begin{figure} % non lo scrivi a mano, ma dai puntini in alto a dx fai instert image
    \centering % mi centra il grafico
    \includegraphics[width=0.3\linewidth]{Screenshot 2024-05-08 alle 17.37.06.png} 
    \includegraphics[width=0.3\linewidth]{Screenshot 2024-05-08 alle 17.37.06.png} \\ % vado a capo
    \includegraphics[width=0.3\linewidth]{Screenshot 2024-05-08 alle 17.37.06.png} 
    \includegraphics[width=0.3\linewidth]{Screenshot 2024-05-08 alle 17.37.06.png} \\ % mi va a capo, se uso bigskip dopo posso anche ometterla
\end{figure}

% piccolo testo per sapere da dove ho preso l'immagine
\bigskip % spazio più ampio
\centering % per centrare il mio testo
\scriptsize{Paganelli et al. (2024)} % per conformità nella presentazione, se metti delle citazioni o usi tutte scriptsize o no
\end{frame}

% questa parte non funziona ma va bene, guarda su Git di duccio
\begin{frame}{Columns} % per mettere di fianco text e immagine

    \begin{columns}
% creo la prima colonna
    \begin{column}{width=0.5\textwidth}
        \includegraphics[width=0.4\linewidth]{Screenshot 2024-05-08 alle 17.37.06.png}       
    \end{column}
% creo la seconda
    \begin{column}{width=0.5\textwidth}
        small text here
    \end{column}
    \end{columns}
\end{frame}

\end{document}
