%Per prima cosa indichiamo il tipo di documento che stiamo facendo e la dimensione del carattere a 12
\documentclass[12pt]{article}

%Poi mettiamo sempre all'inizio i pacchetti che ci serviranno
\usepackage{graphicx} % Ci permette di caricare delle immagini
\usepackage{hyperref} % Ci permette di inserire collegamenti all'interno del testo
\usepackage{natbib} % Ci permette di gestire la bibliografia 
\usepackage{lineno} % Numera le righe del testo
% Il backslash ci indica l'inizio di una funzione \
% Gli argomenti delle funzioni è inserito dentro le parentesi graffe {}
% Le parentesi quadre [] ci indicano modifiche strutturali.

\linenumbers % funzione che numera le righe del testo

% Inizo del mio documento, metto il titolo l'autore e la data

\title{My first LaTeX doc - Chop Suey!}
\author{Beatrice Viola}
\date{10 Maggio 2024} % permette di cambiare l'impostazione della data a seconda degli argomenti che vengono passati. Se non è presente, pesca direttamente la data dal sistema impostandola come mm/gg/aa. ????????

\begin{document} % dopo un begin ci deve essere sempre un end che serve per chiudere l'operazione

\maketitle % Con questa funzione le parti scritte sopra begin vengono inserite nel documento
\tableofcontents % Genera l'indice automaticamente considerando le sezioni che inserisco io
%se poi voglio aggiungere una sezione posso farlo e questa funzione la aggiunge automaticamente

%Facciamo l'abstract
\begin{abstract}
 Wake up (wake up)
Grab a brush and put a little make-up
Hide the scars to fade away the shake-up (hide the scars to fade away the-)
Why'd you leave the keys upon the table?
Here you go create another fable, you wanted to
Grab a brush and put a little make-up, you wanted to
Hide the scars to fade away the shake-up, you wanted to
Why'd you leave the keys upon the table? You wanted to
\end{abstract}
% se non mi piace lo spazio che si crea in automatico all'interno dei paragrafi
%posso usare la funzione \noindent{}

\bigskip % questa funzione serve per lasciare uno spazio più grande, si potrebbe fare anche con \\
% \smallskip mi lascia uno spazio più piccolo

%La funzione \textit mette il testo in corsivo
\textit{Keywords}: Music, Metal, Love % Per mettere le parole chiave del mio articolo

\section{Introduction} % Crea una nuova sezione del documento, l'introduzione in questo caso
\label{sec:intro} % assegno un'etichetta alla sezione così ogni volta che voglio richiamarla posso farlo con la fuznione \ref{} 
% Se sbaglio il nome della sezione compariranno dei ?? nella porzione in cui richiamo la \ref{}
%In questo modo posso valutare molto velocemente se sono presenti errori
%Per togliere il numero della sezione basta bettere un asterisco * prima delle parentesi graffe {}. 

\textbf{I don't think you trust} - \textb mette il suo argomento dentro le graffe in grassetto
In my self-righteous suicide
I cry when angels deserve to die
In my self-righteous suicide
I cry when angels deserve to die

%Citare la bibliografia dentro al testo
\cite{Milana2010} % cito normalmente la mia bibliografia
\citep{Milana2010} % cito la mia bibliografia mettendo sia nome che anno tra parentesi
\citet{LaLoggia2024} % cito la mia bibliografia mettendo solo l'anno tra parentesi mentre il nome fuori
%questo viene fatto ad esempio quando inizio il testo con una citazione, es. Milana (2010) ci disse che i Barbagianni...
\smallskip

\footnote{Grab a brush and put a little make-up} % La funzione per inserire note a piè di pagina

%Per inserire un link dentro al nostro documento uso la funzione \url{}
la canzone la trovi a:
\url{https://www.youtube.com/watch?v=CSvFpBOe8eY&themeRefresh=1} 

%Andiamo alla pagina dopo per essere più ordinati, usiamo la seguente funzion
\newpage{} 

%Nuova sezione: Metodi
\section{Methods} 
%Posso anche creare delle sottosezioni
\subsection{Study Area} % Come l'area di studio
\subsection{Algoritms} % O anche gli algoritmi

% Scriviamo un'equazione
% Sul web LaTeX Math possiamo trovare le diverse funzioni che potrebbero servirci
Prima abbiamo calcolato l'indice di dominanza di Simpson per popolazioni teoricamente infinite:

\ref{eq:somma}: %richiamo l'equazione che vado a creare qui sotto 
\begin{equation} %inzio l'equazione
\lambda= \sum_{i=1}^{S}p_i %\sum inserisce il simbolo sommatoria da i=1 a S (conil trattino basso p_i metto la i al pedice di p)
    \label{eq:somma}: %Creo un'etichetta per la mia equazione che ho richiamato sopra con \ref
\end{equation} %finisco l'equzione

% altra equazione
Poi si è usata la seguente equazione
ref{eq:newton}: 
\begin{equation}
    F = \sqrt[2]{G \frac{m_1\times m_2}{d^2}} % \sqrt per mettere sotto radice,[2] mi indica la potenza della radice, \frac è la frazione
    %con numeratore e denominatore tra parentesi graffe{}
    % l'accento circonflesso ^ serve a mettere d al quadrato
    %\times{} è la funzione per la moltiplicazione
\label{eq:newton}
\end{equation}

\section{Results}

\section{Discussion}
Possiamo discutere i risultati, come abbiamo visto \ref{sec:intro} % richiamo l'etichetta, lable, della sezione Introduzione


%Facciamo un classico elenco puntato con la funzione \begin{itemize}
In questa tesi abbiamo osservato:
\begin{itemize}
    \item Musica
    \item Astronauti
    \item Barbagianni
\end{itemize} % il begin deve sempre essere accompagnato da un end

%Per fare un elenco puntato numerato uso l afunzione \begin{enumerate}
\begin{enumerate} 
    \item Musica
    \item Astronauti
    \item Barbagianni
\end{enumerate}

\newpage % andiamo ad una pagina nuova

%Andiamo ad aggiungere un'immagine
\begin{figure}
    \centering % Per centrare la figura nel foglio
    \includegraphics[width=\textwidth]{inter.png} %[width=\textwidth] l'immagine mi viene grande quanto il testo
%Se volessi l'immagine più pccola del testo metterei width=0.5 ecc...
%per inserire un'immagine questa deve essere stata scaricata sul computer
    \caption{Apodemus sylvaticus} % Didascalia dell'immagine
    
% Creiamo un label sempre per poter richiamare la figura in altri punti del testo
\label{fig:apodemus} 
\end{figure}

\newpage %andiamo alla pagina dopo

% Creiamo la bibliografia
\begin{thebibliography}{999} % è un tipo di bibliografia
%con la funzione \bibitem creo una sorta di label per richiamare la biblio nel testo
%Le parentesi quadre [] mi dicono come appare nel testo quando lo cito
%Le parentesi graffe {} mi dicono il lable che uso per richiamarlo nel testo
\bibitem[Milana, 2010]{Milana2010}  
Aloisi, I., Cai, G., Serafini-Fracassini, D., \& Del Duca, S. (2016). Polyamines in pollen: from microsporogenesis to fertilization. Frontiers in plant science, 7, 180280. % copiamo e incolliamo il secondo punto da cite di G Scholar.
    
    \bibitem[La Loggia, 2024]{LaLoggia2024}
    La Loggia, O. (2024). Developmental Influences on Social Competence and Neuroplasticity: The Impact of Early Social Complexity in a Cooperatively Breeding Fish (Doctoral dissertation, Institute of Ecology).

\end{thebibliography}

\newpage

% facciamo un box
\hline % disegna una horizontal line
\bigskip
lascia uno spazio tra la riga e il testo.
\textbf{Lautaro Martinez}
\bigskip
\hline % chiudo il box disegnando un'altra linea

\end{document} % fondamentale, ci vuole sempre alla fine
