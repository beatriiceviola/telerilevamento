%Per prima cosa indichiamo il tipo di documento che stiamo facendo e la dimensione del carattere a 12
\documentclass[12pt]{article}

%Poi mettiamo sempre all'inizio i pacchetti che ci serviranno
\usepackage{graphicx} % Ci permette di caricare delle immagini
\usepackage{hyperref} % Ci permette di inserire collegamenti all'interno del testo
\usepackage{natbib} % Ci permette di gestire la bibliografia 
\usepackage{lineno} % Numera le righe del testo
% Il backslash ci indica l'inizio di una funzione \
% Gli argomenti delle funzioni è inserito dentro le parentesi graffe {}
% Le parentesi quadre [] ci indicano modifiche strutturali.

\linenumbers % funzione che numera le righe del testo

% Inizo del mio documento, metto il titolo l'autore e la data

\title{My first LaTeX doc - Chop Suey!}
\author{Beatrice Viola}
\date{10 Maggio 2024} % permette di cambiare l'impostazione della data a seconda degli argomenti che vengono passati. Se non è presente, pesca direttamente la data dal sistema impostandola come mm/gg/aa. ????????

\begin{document} % dopo un begin ci deve essere sempre un end che serve per chiudere l'operazione

\maketitle % Con questa funzione le parti scritte sopra begin vengono inserite nel documento
\tableofcontents % Genera l'indice automaticamente considerando le sezioni che inserisco io
%se poi voglio aggiungere una sezione posso farlo e questa funzione la aggiunge automaticamente

%Facciamo l'abstract
\begin{abstract}
 Wake up (wake up)
Grab a brush and put a little make-up
Hide the scars to fade away the shake-up (hide the scars to fade away the-)
Why'd you leave the keys upon the table?
Here you go create another fable, you wanted to
Grab a brush and put a little make-up, you wanted to
Hide the scars to fade away the shake-up, you wanted to
Why'd you leave the keys upon the table? You wanted to
\end{abstract}

\bigskip % questa funzione serve per lasciare uno spazio più grande, si potrebbe fare anche con \\
% \smallskip mi lascia uno spazio più piccolo

\textit{Keywords}: Music, Metal, Love % Per mettere le parole chiave del mio articolo

\section{Introduction} % mi crea una nuova sezione del documento
\label{sec:intro} % assegno un'etichetta alla sezione per richiamarla nel testo con \ref{} - \label viene utilizzato per poter richiamare con \ref
% nel caso in cui si sbagli il nome della sezione compariranno dei ?? nella porzione in cui richiamo la \ref{}. Ciò permette di poter cercare eventuali errori all'interno del testo in modo più semplice.
% per togliere il numero della sezione basta bettere un * prima delle {}. *{}

\textbf{Capitan Zanetti} - \textb mette in grassetto la parte di testo passata come argomento. bold
\textit{è gol è gol è gol} - \textit formatta il testo passato come argomento in corsivo. italic

Vivila Questa storia vivila Può durare una vita
Una sola partita Pazza Inter, amala
E continuerò Nel sole e nel vento la mia festa
E sempre vivrò (e sempre vivrò) Questi colori... nella testa
Neroazzurri Io vi seguirò, neroazzurri
Sempre lì vivrò, nerazzurri Questa mia speranza
È l'essenza, io non vivo senza
Amala Pazza Inter, amala
È una gioia infinita Che dura una vita
Pazza Inter, amala
\cite{Aloisi23} % cito una delle lables della mia bibliografia
\citep{Aloisi23} % citare la biblio tra parentesi
\citet{LaLoggia2024} % citare la biblio come testo con solo l'anno tra parentesi
\smallskip

\footnote{Source: F.C. Internazionale} % per inserire note a pie di pagina

la canzone la trovi a:
\url{https://www.youtube.com/watch?v=2a1Ij1t-ltg} % link alla pagina di riferimento

\section{Methods} % apro una nuova sezione

\subsection{Study Area} % posso creare delle sottosezioni
\subsection{Algoritms}

% Per scrivere un'equazione
% Cerca sul web LaTeX Math per cercare tutte le varie funzioni

Prima abbiamo usato la seguente equazione \ref{eq:sum}: % richiamo l'equazione che creo qui sotto

\begin{equation} % creo l'equazione
    T = \sum p_i % \sum inserisce il simbolo della sommatoria
\label{eq:sum} % Attribuiamo un'etichetta alla eq. per fare i riferimenti
\end{equation} % Fine dell'equazione

% altra equazione
In questa tesi abbiamo usato un'equazione \ref{eq:newton}:
\begin{equation}
    F = \sqrt{G \frac{m_1\times m_2}{d^2}} % \sqrt per mettere sotto radice
    % [] è la potenza della radice. _ serve per il pedice. \times{} è la funzione per la moltiplicazione. \frac{} è la funzione per la frazione.
\label{eq:newton}
\end{equation}

\section{Results}
Milan 1 - 2 Inter, 22 Aprile 2024, 20° scudetto, seconda stella.

\section{Discussion}
I risultati ottenuti, come abbiamo visto nella sezione \ref{sec:intro} % richiamo la lable della sezione into

In questa tesi abbiamo osservato:
\begin{itemize} % Per fare un elenco puntato classico metto come argomento itemize
    \item Inno dell'inter
    \item Pazza Inter
    \item Juve m
\end{itemize} % chiudo sempre il begin che ho iniziato

\begin{enumerate} %Per fare un elenco numerato utilizzo enumerate
    \item Inno dell'inter
    \item Pazza Inter
    \item Juve m
\end{enumerate}

\newpage % andiamo ad una pagina nuova

% aggiungiamo un'immagine

\begin{figure}
    \centering % per centrare la figura nel foglio
    \includegraphics[width=\textwidth]{inter.png} % [] così l'immagine viene larga quanto il testo. Se volessimo farlo la metà, basterebbe scrivere width=0.5, quindi moltiplicare per un fattore di riduzione. L'immagine la devi avere scaricata sul pc.
    \caption{Stemma dell'Inter} % didascalia dell'immagine
    \label{fig:chagall} % sempre per poterla richiamare nel testo
\end{figure}

\newpage
% facciamo la bibliografia
\begin{thebibliography}{999} % è un tipo di bibliografia
    
    \bibitem[Aloisi, 2023]{Aloisi23} % [] come voglio che appaia nel testo, {} è la lable per richiamarlo nel testo.
    Aloisi, I., Cai, G., Serafini-Fracassini, D., \& Del Duca, S. (2016). Polyamines in pollen: from microsporogenesis to fertilization. Frontiers in plant science, 7, 180280. % copiamo e incolliamo il secondo punto da cite di G Scholar.
    
    \bibitem[La Loggia, 2024]{LaLoggia2024}
    La Loggia, O. (2024). Developmental Influences on Social Competence and Neuroplasticity: The Impact of Early Social Complexity in a Cooperatively Breeding Fish (Doctoral dissertation, Institute of Ecology).

\end{thebibliography}

\newpage

% facciamo un box
\hline % disegna una horizontal line
\bigskip
lascia uno spazio tra la riga e il testo.
\textbf{Lautaro Martinez}
\bigskip
\hline % chiudo il box disegnando un'altra linea

\end{document} % fondamentale, ci vuole sempre alla fine
